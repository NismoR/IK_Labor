%----------------------------------------------------------------------------
\appendix
%----------------------------------------------------------------------------
\chapter*{Függelék}\addcontentsline{toc}{chapter}{Függelék}
\setcounter{chapter}{6}  % a fofejezet-szamlalo az angol ABC 6. betuje (F) lesz
%\setcounter{equation}{0} % a fofejezet-szamlalo az angol ABC 6. betuje (F) lesz
%\numberwithin{equation}{section}
\numberwithin{figure}{chapter}
%\numberwithin{lstlisting}{section}
%\numberwithin{tabular}{section}

%----------------------------------------------------------------------------
%\section{A TeXnicCenter felülete}
%----------------------------------------------------------------------------
\setcounter{figure}{0} 
%TODO helyes ábra
%TODO miért 13. ábra?
%\begin{figure}[!ht]
%	\centering
%	\includegraphics[trim = 88mm 86mm 58mm 7.5mm,clip, angle=90, width=140mm,keepaspectratio]{figures/hw-mcu.pdf}
%	\caption{A mikrokontroller lábkiosztása} 
%	\label{fig:McuFig}
%\end{figure}
%\begin{figure}[!ht]
%	\centering
%	\includegraphics[trim = 136mm 7mm 72mm 8mm,clip, width=150mm,keepaspectratio]{figures/sib-pcb-print.png}
%	\caption{A nyomtatott áramkör rajzolata} 
%	\label{fig:PcbPrint}
%\end{figure}
%\begin{figure}[!ht]
%\centering
%\includegraphics[angle=90,clip, width=150mm,keepaspectratio]{figures/sib.jpg}
%\caption{Az egyik elkészült áramkör} 
%\label{fig:Sib}
%\end{figure}


\section*{Második feladat Matlab kódja}
\lstinputlisting[style=Matlab-editor]{figures/m01/fctrl.m}\label{MatlabCode}